\documentclass[fleqn, a4paper, 11pt, oneside]{amsart}
%\usepackage[top = 2cm, bottom = 1cm, left = 1cm, right = 1cm]{geometry}
\usepackage{exsheets, tasks}
\usepackage{amsmath, amssymb, amsthm} %standard AMS packages
\usepackage{marginnote} %marginnotes
\usepackage{gensymb} %miscellaneous symbols
\usepackage{commath} %differential symbols
\usepackage{xcolor} %colours
\usepackage{cancel} %cancelling terms
\usepackage[free-standing-units, space-before-unit]{siunitx} %formatting units
\usepackage{tikz, pgfplots} %diagrams
\usetikzlibrary{calc, hobby, patterns, intersections, decorations.markings}
\usepackage{graphicx} %inserting graphics
\usepackage{hyperref} %hyperlinks
\usepackage{datetime} %date and time
\usepackage{ulem} %underline for \emph{}
\usepackage{xfrac} %inline fractions
\usepackage{enumerate,enumitem} %numbered lists
\usepackage{float} %inserting floats
\usepackage{circuitikz}[american voltages, american currents] %circuit diagrams

\newcommand\numberthis{\addtocounter{equation}{1}\tag{\theequation}} %adds numbers to specific equations in non-numbered list of equations

\newcommand{\AxisRotator}[1][rotate=0]{
	\tikz [x=0.25cm,y=0.60cm,line width=.2ex,-stealth,#1] \draw (0,0) arc (-150:150:1 and 1);%
} %rotation symbols on axes

\theoremstyle{definition}
\newtheorem{example}{Example}
\newtheorem{definition}{Definition}

\theoremstyle{theorem}
\newtheorem{theorem}{Theorem}

\newcommand{\curl}{\mathrm{curl\,}}

\makeatletter
\@addtoreset{section}{part} %resets section numbers in new part
\makeatother

\renewcommand{\thesubsection}{(\arabic{subsection})}
\renewcommand{\thesection}{(\arabic{section})}

%section headings on left
\makeatletter
\def\specialsection{\@startsection{section}{1}%
	\z@{\linespacing\@plus\linespacing}{.5\linespacing}%
	%  {\normalfont\centering}}% DELETED
	{\normalfont}}% NEW
\def\section{\@startsection{section}{1}%
	\z@{.7\linespacing\@plus\linespacing}{.5\linespacing}%
	%  {\normalfont\scshape\centering}}% DELETED
	{\normalfont\scshape}}% NEW
\makeatother

%forces newline after subsection
\makeatletter
\def\subsection{\@startsection{subsection}{3}%
	\z@{.5\linespacing\@plus.7\linespacing}{.1\linespacing}%
	{\normalfont\itshape}}
\makeatother

\settasks{counter-format = tsk[1].}

\SetupExSheets{solution/print = true}

%opening
\title{Introduction to Linear Systems : Assignment 2}
\author
{
	Aakash Jog\\
	ID : 989323563
}
\date{\formatdate{29}{10}{2015}}

\begin{document}

\tikzset{->-/.style={decoration={
  markings,
  mark=at position #1 with {\arrow{>}}},postaction={decorate}}}

\maketitle
%\setlength{\mathindent}{0pt}

\begin{question}
	The transfer function of a system is given by
	\begin{align*}
		G(s) & = \frac{s + 2}{s^3 + 8 s^2 + 19 s + 12}
	\end{align*}
	The initial conditions of the system are unknowns denoted by
	\begin{align*}
		y(0^-)   & = a \\
		y'(0^-)  & = b \\
		y''(0^-) & = c
	\end{align*}
	\begin{enumerate}
		\item
			Find the ODE that represents the system.
		\item
			Find the initial conditions of the system, i.e. find $a$, $b$, $c$, and the missing $\lambda_1$, $\lambda_3$, if it is known that the homogeneous solution of the system is
			\begin{align*}
				y_h(t) & = \left( e^{\lambda_1 t} + 3 e^{-3 t} + 4 e^{\lambda_3 t} \right) \delta_{-1}(t)
			\end{align*}
			such that $|\lambda_1| < |\lambda_3|$.
		\item
			Find the full response of the system to a unit step input.
	\end{enumerate}
\end{question}

\begin{solution}
	\begin{enumerate}[leftmargin=*]
		\item
			\begin{align*}
				G(s)                                                           & = \frac{s + 2}{s^3 + 8 s^2 + 19 s + 12} \\
				\therefore \frac{Y(s)}{U(s)}                                   & = \frac{s + 2}{s^3 + 8 s^2 + 19 s + 12} \\
				\therefore Y(s) \left( s^3 + 8 s^2 + 19 s + 12 \right)         & = U(s) (s + 2)                          \\
				\therefore y^{(3)}(t) + 8 y^{(2)}(t) + 19 y^{(1)}(t) + 12 y(t) & = u^{(1)}(t) + 2 u(t)
			\end{align*}
		\item
			The corresponding homogeneous ODE is
			\begin{align*}
				y^{(3)}(t) + 8 y^{(2)}(t) + 19 y^{(1)}(t) + 12 y(t) & = 0
			\end{align*}
			Therefore, the characteristic equation is
			\begin{align*}
				\lambda^3 + 8 \lambda^2 + 19 \lambda + 12            & = 0 \\
				\therefore (\lambda + 3) (\lambda^2 + 5 \lambda + 4) & = 0 \\
				\therefore (\lambda + 3) (\lambda + 1) (\lambda + 4) & = 0
			\end{align*}
			Therefore,
			\begin{align*}
				\lambda_1 & = -1 \\
				\lambda_2 & = -3 \\
				\lambda_3 & = -4
			\end{align*}
			Therefore,
			\begin{align*}
				y_h(t)                    & = e^{-t} + 3 e^{-3 t} + 4 e^{-4 t}   \\
				\therefore {y_h}^{(1)}(t) & = -e^{-t} - 9 e^{-3 t} - 16 e^{-4 t} \\
				\therefore {y_h}^{(2)}(t) & = e^{-t} + 27 e^{-3 t} + 64 e^{-4 t}
			\end{align*}
			Therefore,
			\begin{align*}
				y_h(0)         & = 1 + 3 + 4   \\
				\therefore a   & = 8           \\
				{y_h}^{(1)}(0) & = -1 - 9 - 16 \\
				\therefore b   & = -26         \\
				{y_h}^{(2)}(0) & = 1 + 27 + 64 \\
				\therefore c   & = 92
			\end{align*}
		\item
			\begin{align*}
				Y_p(s) & = G(s) U(s) \\
                       & = \frac{s + 2}{(s + 1) (s + 3) (s + 4)} U(s)
			\end{align*}
			Therefore, as $u(t) = \delta_{-1}(t)$,
			\begin{align*}
				Y_p(s) & = \frac{s + 1}{(s + 1) (s + 3) (s + 4)} \frac{1}{s} \\
                       & = \frac{s + 1}{s (s + 1) (s + 3) (s + 4)}
			\end{align*}
			Therefore, let
			\begin{align*}
				Y(s) & = \frac{A}{s} + \frac{B}{s + 1} + \frac{C}{s + 3} + \frac{D}{s + 4}
			\end{align*}
			Therefore,
			\begin{align*}
				A & = \left. \frac{s + 2}{s (s + 1) (s + 3) (s + 4)} \right|_{s = 0} \\
                  & = \frac{1}{6}                                                    \\
				B & = \left. \frac{s + 2}{s (s + 3) (s + 4)} \right|_{s = 1}         \\
                  & = -\frac{1}{6}                                                   \\
				C & = \left. \frac{s + 2}{s (s + 1) (s + 4)} \right|_{s = 3}         \\
                  & = -\frac{1}{6}                                                   \\
				D & = \left. \frac{s + 2}{s (s + 1) (s + 3)} \right|_{s = 4}         \\
                  & = \frac{1}{6}
			\end{align*}
			Therefore,
			\begin{align*}
				Y_p(s) & = \frac{1}{6 s} - \frac{1}{6 (s + 1)} - \frac{1}{6 (s + 3)} + \frac{1}{6 (s + 4)} \\
				y_p(t) & = \left( \frac{1}{6} - \frac{1}{6} e^{-t} - \frac{1}{6} e^{-3 t} + \frac{1}{6} e^{-4 t} \right) \delta_{-1}(t)
			\end{align*}
			Therefore,
			\begin{align*}
				y(t) & = y_h(t) + y_p(t)                                                                                                                                                               \\
                     & = \left( e^{-t} + 3 e^{-3 t} + 4 e^{-4 t} \right) \delta_{-1}(t) + \left( \frac{1}{6} - \frac{1}{6} e^{-t} - \frac{1}{6} e^{-3 t} + \frac{1}{6} e^{-4 t} \right) \delta_{-1}(t) \\
                     & = \left( \frac{1}{6} + \frac{5}{6} e^{-t} + \frac{17}{6} e^{-3 t} + \frac{25}{6} e^{-4 t} \right) \delta_{-1}(t)
			\end{align*}
	\end{enumerate}
\end{solution}

\begin{question}
	The following ODE is given.
	\begin{align*}
		y'''(t) + 6 y''(t) + 11 y'(t) + 6 y(t) & = a u'(t) + b u(t) \\
		y(0^-)                                 & = -2               \\
		y'(0^-)                                & = 1                \\
		y''(0^-)                               & = 0
	\end{align*}
	\begin{enumerate}
		\item
			Find the system's response to the initial conditions.
			Hint: One of the characteristic polynomial zeros is at $\lambda = -2$.
		\item
			It is given that the system's full response to the input $u(t) = 2 e^{-4 t} \delta_{-1}(t)$, and the initial conditions is
			\begin{align*}
				y(t) & = \left( -\frac{5}{2} e^{-t} + 3 e^{-2 t} - \frac{11}{2} e^{-3 t} + 3 e^{-4 t} \right) \delta_{-1}(t)
			\end{align*}
			Find
			\begin{enumerate}
				\item the values of $a$ and $b$
				\item the system's transfer function
			\end{enumerate}
		\item
			Find the total response, including initial conditions, to the input $f(t) = 3 e^{-2 t} \delta_{-1}(t)$.
	\end{enumerate}
\end{question}

\begin{solution}
	\begin{enumerate}[leftmargin=*]
		\item
			\begin{align*}
				y'''(t) + 6 y''(t) + 11 y'(t) + 6 y(t) & = a u'(t) + b u(t) \\
			\end{align*}
			Therefore, the corresponding homogeneous ODE is
			\begin{align*}
				y'''(t) + 6 y''(t) + 11 y'(t) + 6 y(t) & = 0
			\end{align*}
			Therefore, the characteristic equation is
			\begin{align*}
				\lambda^3 + 6 \lambda^2 + 11 \lambda + 6             & = 0 \\
				\therefore (\lambda + 2) (\lambda^2 + 4 \lambda + 3) & = 0 \\
				\therefore (\lambda + 2) (\lambda + 1) (\lambda + 3) & = 0
			\end{align*}
			Therefore,
			\begin{align*}
				\lambda_1 & = -1 \\
				\lambda_2 & = -2 \\
				\lambda_3 & = -3
			\end{align*}
			\begin{align*}
				y_h(t)                & = A e^{-t} + B e^{-2 t} + C e^{-3 t}        \\
				\therefore {y_h}'(t)  & = -A e^{-t} + -2 B e^{-2 t} + -3 C e^{-3 t} \\
				\therefore {y_h}''(t) & = A e^{-t} + 4 B e^{-2 t} + 9 C e^{-3 t}
			\end{align*}
			Therefore, substituting in initial conditions,
			\begin{align*}
				-2 & = A + B + C      \\
				1  & = -A - 2 B - 3 C \\
				0  & = A + 4 B + 9 C
			\end{align*}
			Therefore, solving,
			\begin{align*}
				A & = -\frac{7}{2} \\
				B & = 2            \\
				C & = -\frac{1}{2}
			\end{align*}
			Therefore,
			\begin{align*}
				y_h(t) & = \left( -\frac{7}{2} e^{-t} + 2 e^{-2 t} - \frac{1}{2} e^{-3 t} \right) \delta_{-1}(t)
			\end{align*}
		\item
			\begin{enumerate}[leftmargin=*]
				\item
					Taking the Laplace transform of the ODE,
					\begin{align*}
						Y_p(s) & = \frac{a s + b}{s^3 + 6 s^2 + 11 s + 6} U(s) \\
                               & = \frac{2 (a s + b)}{(s + 1) (s + 2) (s + 3) (s + 4)}
					\end{align*}
					Let
					\begin{align*}
						Y_p(s) & = \frac{A}{s + 1} + \frac{B}{s + 2} + \frac{C}{s + 3} + \frac{D}{s + 4}
					\end{align*}
					Therefore,
					\begin{align*}
						A & = \left. \frac{2 (a s + b)}{(s + 2) (s + 3) (s + 4)} \right|_{s = -1} \\
                          & = \frac{b - a}{3}                                                     \\
						B & = \left. \frac{2 (a s + b)}{(s + 1) (s + 3) (s + 4)} \right|_{s = -2} \\
                          & = 2 a - b                                                             \\
						C & = \left. \frac{2 (a s + b)}{(s + 1) (s + 2) (s + 4)} \right|_{s = -3} \\
                          & = b - 3 a                                                             \\
						D & = \left. \frac{2 (a s + b)}{(s + 1) (s + 2) (s + 3)} \right|_{s = -4} \\
                          & = \frac{4 a - b}{3}
					\end{align*}
					Comparing to the given response,
					\begin{align*}
						1  & = A               \\
                           & = \frac{b - a}{3} \\
						1  & = B               \\
                           & = 2 a - b         \\
						-5 & = C               \\
                           & = b - 3 a         \\
						3  & = D               \\
                           & = \frac{4 a - b}{3}
					\end{align*}
					Therefore, solving,
					\begin{align*}
						a & = 4 \\
						b & = 7
					\end{align*}
				\item
					If the RHS is $\delta(t)$, the ODE is
					\begin{align*}
						{y_{\delta}}'''(t) + 6 {y_{\delta}}''(t) + 11 {y_{\delta}}'(t) + 6 y_{\delta}(t) & = 0 \\
						y_{\delta}(0^+)                                                                  & = 0 \\
						{y_{\delta}}'(0^+)                                                               & = 0 \\
						{y_{\delta}}''(0^+)                                                              & = 1
					\end{align*}
					Therefore,
					\begin{align*}
						y_{\delta}(t)                & = A e^{-t} + B e^{-2 t} + C e^{-3 t}      \\
						\therefore {y_{\delta}}'(t)  & = -A e^{-t} - 2 B e^{-2 t} - 3 C e^{-3 t} \\
						\therefore {y_{\delta}}''(t) & = A e^{-t} + 4 B e^{-2 t} + 9 C e^{-3 t}
					\end{align*}
					Therefore,
					\begin{align*}
						0 & = A + B + C       \\
						0 & = - A - 2 B - 3 C \\
						1 & = A + 4 B + 9 C
					\end{align*}
					Therefore, solving,
					\begin{align*}
						A & = \frac{1}{2} \\
						B & = -1          \\
						C & = \frac{1}{2}
					\end{align*}
					Therefore,
					\begin{align*}
						y_{\delta}(t) & = \left( \frac{1}{2} e^{-t} - e^{-2 t} + \frac{1}{2} e^{-3 t} \right) \delta_{-1}(t)
					\end{align*}
					Therefore, the impulse response is
					\begin{align*}
						g(t) & = 4 {y_{\delta}}'(t) + 7 y_{\delta}(t) \\
                             & = \left( \frac{3}{2} e^{-t} + e^{-2 t} - \frac{5}{2} e^{-3 t} \right) \delta_{-1}(t)
					\end{align*}
					Therefore,
					\begin{align*}
						G(s) & = \frac{3}{2} \frac{1}{s + 1} + \frac{1}{s + 2} - \frac{5}{2} \frac{1}{s + 3} \\
                             & = \frac{4 s + 7}{s^3 + 6 s^2 + 11 s + 6}
					\end{align*}
			\end{enumerate}
		\item
			\begin{align*}
				u(t) & = 3 e^{-2 t} \delta_{-1}(t)
			\end{align*}
			Therefore,
			\begin{align*}
				U(s) & = \frac{3}{s + 2}
			\end{align*}
			Therefore,
			\begin{align*}
				Y(s) & = \frac{-2 s^2 - 11 s - 16}{(s + 1) (s + 2) (s + 3)} + 3 \frac{4 s + 7}{(s + 1) (s + 2)^2 (s + 3)} \\
                     & = \frac{-2 s^3 - 15 s^2 - 26 s - 11}{(s + 1) (s + 2)^2 (s + 3)}
			\end{align*}
			Let
			\begin{align*}
				Y(s) & = \frac{A}{s + 1} + \frac{B}{s + 2} + \frac{C}{(s + 2)^2} + \frac{D}{s + 3}
			\end{align*}
			Therefore, solving,
			\begin{align*}
				A & = 1   \\
				B & = -10 \\
				C & = 3   \\
				D & = 7
			\end{align*}
			Therefore,
			\begin{align*}
				Y(s) & = \frac{1}{s + 1} - \frac{10}{s + 2} + \frac{3}{(s + 2)^2} + \frac{7}{s + 3}
			\end{align*}
			Therefore,
			\begin{align*}
				y(t) & = \left( e^{-t} - 10 e^{-2 t} + 3 t e^{-2 t} + 7 e^{-3 t} \right) \delta_{-1}(t)
			\end{align*}
	\end{enumerate}
\end{solution}

\begin{question}
	The system's input response (particular solution) to a unit step response is
	\begin{align*}
		y_{-1}(t) & = \left( \frac{1}{2} e^{-t} - \frac{1}{2} e^{-3 t} \right) \delta_{-1}(t)
	\end{align*}
	\begin{enumerate}
		\item
			Find the system's transfer function.
			Find the ODE that represents the system.
		\item
			Find the final value of the system's response to the input
			\begin{align*}
				f(t) & = \left( 3 e^{-t} \cos(t) + \frac{1}{2} \right) \delta_{-1}(t)
			\end{align*}
		\item
			A system is described by the following ODE.
			\begin{align*}
				y'(t) + a y(t) + b u(t)
			\end{align*}
			It is given that for that the total response for the input
			\begin{align*}
				f(t) & = 2 e^{-4 t} \delta_{-1}(t)
			\end{align*}
			is
			\begin{align*}
				y(t) & = \left( 3 e^{2 t} - e^{-4 t} \right) \delta_{-1}(t)
			\end{align*}
			Find the system's transfer function.
			Find the initial condition of the system, i.e. $y(0)$.
	\end{enumerate}
\end{question}

\begin{solution}
	\begin{enumerate}[leftmargin=*]
		\item
			\begin{align*}
				y_{-1}(t) & = \left( \frac{1}{2} e^{-t} - \frac{1}{2} e^{-3 t} \right) \delta_{-1}(t)
			\end{align*}
			Therefore,
			\begin{align*}
				g(t) & = \dod{}{t}\left( y_{-1}(t) \right)                                                                                                             \\
                     & = \dod{}{t}\left( \left( \frac{1}{2} e^{-t} - \frac{1}{2} e^{-3 t} \right) \delta_{-1}(t) \right)                                               \\
                     & = \left( -\frac{1}{2} e^{-t} + \frac{3}{2} e^{-3 t} \right) \delta_{-1}(t) + \left( \frac{1}{2} e^{-t} - \frac{1}{2} e^{-3 t} \right) \delta(t) \\
                     & = \left( -\frac{1}{2} e^{-t} + \frac{3}{2} e^{-3 t} \right) \delta_{-1}(t)
			\end{align*}
			Therefore,
			\begin{align*}
				G(s) & = -\frac{1}{2} \frac{1}{s + 1} + \frac{3}{2} \frac{1}{s + 3} \\
                     & = \frac{s}{(s + 1) (s + 3)}
			\end{align*}
			Therefore,
			\begin{align*}
				Y(s) \left( s^2 + 4 s + 3 \right) & = s U(s)
			\end{align*}
			Therefore, the ODE is
			\begin{align*}
				y''(t) + 4 y'(t) + 3 y(t) & = u'(t)
			\end{align*}
		\item
			\begin{align*}
				u(t)            & = \left( 3 e^{-t} \cos(t) + \frac{1}{2} \right) \delta_{-1}(t) \\
				\therefore U(s) & = \frac{3 (s + 1)}{(s + 1)^2 + 1} + \frac{1}{2 s}              \\
                                & = \frac{7 s^2 + 8 s + 2}{2 s \left( s^2 + 2 s + 2 \right)}
			\end{align*}
			Therefore,
			\begin{align*}
				Y(s) & = G(s) U(s) \\
                     & = \frac{s}{s^2 + 4 s + 3} \frac{7 s^2 + 8 s + 2}{2 s \left( s^2 + 2 s + 2 \right)}
			\end{align*}
			Therefore,
			\begin{align*}
				\lim\limits_{t \to \infty} y(t) & = \lim\limits_{s \to 0} s Y(s)                                                                           \\
                                                & = \lim\limits_{s \to 0} \frac{s}{s^2 + 4 s + 3} \frac{7 s^2 + 8 s + 2}{2 s \left( s^2 + 2 s + 2 \right)} \\
                                                & = 0
			\end{align*}
		\item
			\begin{align*}
				y'(t) + a y(t) & = b u(t)
			\end{align*}
			Therefore,
			\begin{align*}
				s Y(s) - y(0^-) + a Y(s) & = b U(s)          \\
				\therefore Y(s) (s + a)  & = b U(s) + y(0^-) \\
				\therefore Y(s)          & = \frac{b}{s + a} U(s) + \frac{y(0^-)}{s + a}
			\end{align*}
			According to the given data,
			\begin{align*}
				y(t)            & = \left( 3 e^{2 t} - e^{-4 t} \right) \delta_{-1}(t) \\
				\therefore Y(s) & = \frac{3}{s - 2} - \frac{1}{s + 4}                  \\
				u(t)            & = \left( 2 e^{-4 t} \right) \delta_{-1}(t)           \\
				\therefore U(s) & = \frac{2}{s + 4}
			\end{align*}
			Therefore, comparing and solving,
			\begin{align*}
				a      & = -2 \\
				b      & = 3  \\
				y(0^-) & = 2
			\end{align*}
			Therefore,
			\begin{align*}
				G(s) & = \left. \left( \frac{Y(s)}{U(s)} \right) \right|_{y(0^-) = 0} \\
                     & = \frac{b}{s + a}                                              \\
                     & = \frac{3}{s - 2}
			\end{align*}
	\end{enumerate}
\end{solution}

\begin{question}
	A system's input response (particular solution) is given
	\begin{align*}
		y_{-1}(t) & = \left( 1 + e^{-3 t} \sin(t) - e^{-3 t} \cos(t) \right) \delta_{-1}(t)
	\end{align*}
	\begin{enumerate}
		\item
			Find the system's transfer function.
		\item
			Find the ODE that represents the system.
		\item
			Find the system's input response (particular solution) to the input
			\begin{align*}
				f(t) & = e^{-4 t} \delta_{-1}(t)
			\end{align*}
			assuming zero initial conditions.
	\end{enumerate}
\end{question}

\begin{question}
	The transfer function of a system is given by
	\begin{align*}
		G(s) & = \frac{s + 3}{(s + 1) (s + 2)^2 (s - 4)}
	\end{align*}
	Find the system's impulse response $g(t)$ using the Residue Theorem.
\end{question}

\begin{solution}
	\begin{align*}
		G(s) & = \frac{s + 3}{(s + 1) (s + 2)^2 (s - 4)}
	\end{align*}
	Therefore,
	\begin{align*}
		g(t) & = \sum\limits_{k = 1}^{3} \textrm{Residue}\left( \frac{s + 3}{(s + 1) (s + 2)^2 (s - 4)} e^{s t} , \lambda_k \right)
	\end{align*}
	Therefore,
	\begin{align*}
		X_1(s)             & = \frac{s + 3}{(s + 2)^2 (s - 4)} \\
		\therefore X_1(-1) & = -\frac{2}{5} e^{-t}             \\
		X_3(s)             & = \frac{s + 3}{(s + 1) (s + 2)^2}
	\end{align*}
	The residue at $\lambda_2 = 2$ is
	\begin{align*}
		\lim\limits_{s \to -2} \frac{1}{1!} \dod{}{s}\left( \frac{(s + 3) e^{s t}}{(s + 1) (s - 4)} \right) & = \frac{1}{6} t e^{-2 t} + \frac{7}{36} e^{-2 t}
	\end{align*}
	Therefore,
	\begin{align*}
		g(t) & = -\frac{2}{5} e^{-t} + \frac{1}{6} t e^{-2 t} + \frac{7}{36} e^{-2 t} + \frac{7}{180} e^{4 t}
	\end{align*}
\end{solution}

\end{document}
