\documentclass[fleqn, a4paper, 11pt, oneside]{amsart}
%\usepackage[top = 2cm, bottom = 1cm, left = 1cm, right = 1cm]{geometry}
\usepackage{exsheets, tasks}
\usepackage{amsmath, amssymb, amsthm} %standard AMS packages
\usepackage{marginnote} %marginnotes
\usepackage{gensymb} %miscellaneous symbols
\usepackage{commath} %differential symbols
\usepackage{xcolor} %colours
\usepackage{cancel} %cancelling terms
\usepackage[free-standing-units, space-before-unit]{siunitx} %formatting units
\usepackage{tikz, pgfplots} %diagrams
\usetikzlibrary{calc, hobby, patterns, intersections, decorations.markings}
\usepackage{graphicx} %inserting graphics
\usepackage{hyperref} %hyperlinks
\usepackage{datetime} %date and time
\usepackage{ulem} %underline for \emph{}
\usepackage{xfrac} %inline fractions
\usepackage{enumerate,enumitem} %numbered lists
\usepackage{float} %inserting floats
\usepackage{circuitikz}[american voltages, american currents] %circuit diagrams

\newcommand\numberthis{\addtocounter{equation}{1}\tag{\theequation}} %adds numbers to specific equations in non-numbered list of equations

\newcommand{\AxisRotator}[1][rotate=0]{
	\tikz [x=0.25cm,y=0.60cm,line width=.2ex,-stealth,#1] \draw (0,0) arc (-150:150:1 and 1);%
} %rotation symbols on axes

\theoremstyle{definition}
\newtheorem{example}{Example}
\newtheorem{definition}{Definition}

\theoremstyle{theorem}
\newtheorem{theorem}{Theorem}

\newcommand{\curl}{\mathrm{curl\,}}

\makeatletter
\@addtoreset{section}{part} %resets section numbers in new part
\makeatother

\renewcommand{\thesubsection}{(\arabic{subsection})}
\renewcommand{\thesection}{(\arabic{section})}

%section headings on left
\makeatletter
\def\specialsection{\@startsection{section}{1}%
	\z@{\linespacing\@plus\linespacing}{.5\linespacing}%
	%  {\normalfont\centering}}% DELETED
	{\normalfont}}% NEW
\def\section{\@startsection{section}{1}%
	\z@{.7\linespacing\@plus\linespacing}{.5\linespacing}%
	%  {\normalfont\scshape\centering}}% DELETED
	{\normalfont\scshape}}% NEW
\makeatother

%forces newline after subsection
\makeatletter
\def\subsection{\@startsection{subsection}{3}%
	\z@{.5\linespacing\@plus.7\linespacing}{.1\linespacing}%
	{\normalfont\itshape}}
\makeatother

\settasks{counter-format = tsk[1].}

\SetupExSheets{solution/print = true}

%opening
\title{Introduction to Linear Systems : Assignment 5a}
\author
{
	Aakash Jog\\
	ID : 989323563
}
\date{\formatdate{19}{11}{2015}}

\begin{document}

\tikzset{->-/.style={decoration={
  markings,
  mark=at position #1 with {\arrow{>}}},postaction={decorate}}}

\maketitle
%\setlength{\mathindent}{0pt}

\begin{question}
	A mass $M_1$ is placed on a dashpot having a damping coefficient $B_1$.
	The dashpot at its other side is connected to a mass $M_2$.
	The latter is lying on a dashpot with damping coefficient $B_2$.
	The system is situated in a gravity field $g$.
	\begin{enumerate}
		\item
			Draw an equivalent electrical diagram and write a set of differential equations of order 1, for the velocities $v_1$, $v_2$ of the two masses.
		\item
			Arrange your above solution in a state space presentation with the state vector
			\begin{align*}
				x(t) &=
					\begin{pmatrix}
						v_1 \\
						v_2 \\
					\end{pmatrix}
			\end{align*}
		\item
			Find the suitable output matrix $C$ for the following cases.
			\begin{enumerate}
				\item The output of the system is the average of the two velocities.
				\item The output of the system is the difference between the two velocities.
			\end{enumerate}
	\end{enumerate}
\end{question}

\begin{solution}
	\begin{enumerate}[leftmargin=*]
		\item
			\begin{figure}[H]
				\centering
				\begin{circuitikz}
					\draw (0,0) to [american current source = $m_1 g$] (0,4);
					\draw (2,0) to [C = $m_1$] (2,4);
					\draw (4,0) to [C = $m_1$] (4,4);
					\draw (6,0) to [american current source = $m_2 g$] (6,4);
					\draw (8,0) to [R = $\frac{1}{B_2}$] (8,4);

					\draw (0,4) to (2,4);
					\draw (2,4) to [R = $\frac{1}{B_1}$] (4,4);
					\draw (4,4) to (8,4);

					\draw (0,0) to (8,0);

					\draw (4,0) node [ground] {};
				\end{circuitikz}
			\end{figure}
			\begin{align*}
				{v_1}' & = -\frac{B_1}{m_1} v_1 + \frac{B_1}{m_1} v_2 + g \\
				{v_2}' & = \frac{B_2}{m_2} v_1 - \frac{B_1 + B_2}{m_2} v_2 + g
			\end{align*}
		\item
			\begin{align*}
				{v_1}' & = -\frac{B_1}{m_1} v_1 + \frac{B_1}{m_1} v_2 + g \\
				{v_2}' & = \frac{B_2}{m_2} v_1 - \frac{B_1 + B_2}{m_2} v_2 + g
			\end{align*}
			Therefore,
			\begin{align*}
					\begin{pmatrix}
						{v_1}' \\
						{v_2}' \\
					\end{pmatrix}
				&=
					\begin{pmatrix}
						-\frac{B_1}{m_1} & \frac{B_1}{m_2}        \\
						\frac{B_2}{m_2}  & -\frac{B_1 + B_2}{m_2} \\
					\end{pmatrix}
					\begin{pmatrix}
						v_1 \\
						v_2 \\
					\end{pmatrix}
				+
					\begin{pmatrix}
						1 \\
						1 \\
					\end{pmatrix}
					g
			\end{align*}
		\item
			\begin{enumerate}[leftmargin=*]
				\item
					\begin{align*}
						C &=
							\begin{pmatrix}
								\frac{1}{2} & \frac{1}{2} \\
							\end{pmatrix}
					\end{align*}
				\item
					If $v_1 \ge v_2$,
					\begin{align*}
						C &=
							\begin{pmatrix}
								1 & -1 \\
							\end{pmatrix}
					\end{align*}
				\item
					If $v_2 \ge v_1$,
					\begin{align*}
						C &=
							\begin{pmatrix}
								-1 & 1 \\
							\end{pmatrix}
					\end{align*}
			\end{enumerate}
	\end{enumerate}
\end{solution}

\end{document}
