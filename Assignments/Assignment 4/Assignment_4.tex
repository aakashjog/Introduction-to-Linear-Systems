\documentclass[fleqn, a4paper, 11pt, oneside]{amsart}
%\usepackage[top = 2cm, bottom = 1cm, left = 1cm, right = 1cm]{geometry}
\usepackage{exsheets, tasks}
\usepackage{amsmath, amssymb, amsthm} %standard AMS packages
\usepackage{marginnote} %marginnotes
\usepackage{gensymb} %miscellaneous symbols
\usepackage{commath} %differential symbols
\usepackage{xcolor} %colours
\usepackage{cancel} %cancelling terms
\usepackage[free-standing-units, space-before-unit]{siunitx} %formatting units
\usepackage{tikz, pgfplots} %diagrams
\usetikzlibrary{calc, hobby, patterns, intersections, decorations.markings}
\usepackage{graphicx} %inserting graphics
\usepackage{hyperref} %hyperlinks
\usepackage{datetime} %date and time
\usepackage{ulem} %underline for \emph{}
\usepackage{xfrac} %inline fractions
\usepackage{enumerate,enumitem} %numbered lists
\usepackage{float} %inserting floats
\usepackage{circuitikz}[american voltages, american currents] %circuit diagrams

\newcommand\numberthis{\addtocounter{equation}{1}\tag{\theequation}} %adds numbers to specific equations in non-numbered list of equations

\newcommand{\AxisRotator}[1][rotate=0]{
	\tikz [x=0.25cm,y=0.60cm,line width=.2ex,-stealth,#1] \draw (0,0) arc (-150:150:1 and 1);%
} %rotation symbols on axes

\theoremstyle{definition}
\newtheorem{example}{Example}
\newtheorem{definition}{Definition}

\theoremstyle{theorem}
\newtheorem{theorem}{Theorem}

\newcommand{\curl}{\mathrm{curl\,}}

\makeatletter
\@addtoreset{section}{part} %resets section numbers in new part
\makeatother

\renewcommand{\thesubsection}{(\arabic{subsection})}
\renewcommand{\thesection}{(\arabic{section})}

%section headings on left
\makeatletter
\def\specialsection{\@startsection{section}{1}%
	\z@{\linespacing\@plus\linespacing}{.5\linespacing}%
	%  {\normalfont\centering}}% DELETED
	{\normalfont}}% NEW
\def\section{\@startsection{section}{1}%
	\z@{.7\linespacing\@plus\linespacing}{.5\linespacing}%
	%  {\normalfont\scshape\centering}}% DELETED
	{\normalfont\scshape}}% NEW
\makeatother

%forces newline after subsection
\makeatletter
\def\subsection{\@startsection{subsection}{3}%
	\z@{.5\linespacing\@plus.7\linespacing}{.1\linespacing}%
	{\normalfont\itshape}}
\makeatother

\settasks{counter-format = tsk[1].}

\SetupExSheets{solution/print = true}

%opening
\title{Introduction to Linear Systems : Assignment 4}
\author
{
	Aakash Jog\\
	ID : 989323563
}
\date{\formatdate{12}{11}{2015}}

\begin{document}

\tikzset{->-/.style={decoration={
  markings,
  mark=at position #1 with {\arrow{>}}},postaction={decorate}}}

\maketitle
%\setlength{\mathindent}{0pt}

\begin{question}
	Consider the system in the following illustration.
	\begin{figure}[H]
		\centering
	\end{figure}<++>
	The positive direction is in the direction of the moment $\tau$ that operates on the system.
	The torsional stiffness $k_2$ operates between $B_1$ and $B_2$ only.
	The segment between $B_2$ and $B_3$ is completely stiff.\\
	A torsional damping with damping coefficient $B_1$ is located next to the moment of inertia $J_1$.
	\begin{enumerate}
		\item Draw an equivalent electrical diagram, and mark the angular velocity $\omega_1$ of the moment of inertia $J_1$, and the angular velocity $\omega_2$ of the stiff axle between $B_2$ and $B_3$.
		\item Write the matrix set of equations using Kirchoff's Junction Rule.
		\item Find the transfer function between the angular velocity $\omega_1$ of the moment of inertia $J_1$ and the moment $\tau$ that operates on the system.
	\end{enumerate}
\end{question}

\begin{solution}
	\begin{enumerate}[leftmargin=*]
		\item
			~\\
			\begin{figure}[H]
				\centering
				\begin{circuitikz}
					\draw (0,0) to [R = $\frac{1}{B_3}$] (0,4);
					\draw (2,0) to [american current source = $T$] (2,4);
					\draw (4,0) to [R = $\frac{1}{B_2}$] (4,4);
					\draw (6,0) to [R = $\frac{1}{B_1}$] (6,4);
					\draw (8,0) to [C = $J_1$] (8,4);
					\draw (10,0) to (10,4);
					
					\draw (0,4) to (4,4);
					\draw (4,4) to [L = $\frac{1}{k_2}$] (6,4);
					\draw (6,4) to (8,4);
					\draw (8,4) to [L = $\frac{1}{k_1}$] (10,4);

					\draw (0,0) to (10,0);

					\draw (5,0) node [ground] {};

					\filldraw (2,4) circle (1pt) node [above] {$\omega_2$};
					\filldraw (8,4) circle (1pt) node [above] {$\omega_1$};
				\end{circuitikz}
			\end{figure}
		\item
			\begin{align*}
					\begin{pmatrix}
						\frac{k_1}{s} + \frac{k_2}{s} + B_1 + s J_1 & \frac{-k_2}{s}\\
						-\frac{k_2}{s} & \frac{k_2}{s} + B_2 + B_3\\
					\end{pmatrix}
					\begin{pmatrix}
						\omega_1(s)\\
						\omega_2(s)\\
					\end{pmatrix}
				&=
					\begin{pmatrix}
						0\\
						\tau(s)\\
					\end{pmatrix}
			\end{align*}
		\item
			By Cramer's Rule,
			\begin{align*}
				\omega_1(s) &= \frac{(0) \left( \frac{k_2}{s} + B_2 + B_3 \right) + \tau(s) \left( \frac{k_2}{s} \right)}{\left( \frac{k_1}{s} + \frac{k_2}{s} + B_1 + s J_1 \right) \left( \frac{k_2}{s} + B_2 + B_3 \right) - \left( \frac{k_2}{s} \right)^2}\\
				\therefore \frac{\omega_1(s)}{\tau(s)} &= 
				\frac{\frac{k_2}{s}}{\left( \frac{k_1}{s} + \frac{k_2}{s} + B_1 + s J_1 \right) \left( \frac{k_2}{s} + B_2 + B_3 \right) - \left( \frac{k_2}{s} \right)^2}\\
			\end{align*}
	\end{enumerate}
\end{solution}

\end{document}
