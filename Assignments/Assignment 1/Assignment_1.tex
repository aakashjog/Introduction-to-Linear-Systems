\documentclass[fleqn, a4paper, 11pt, oneside]{amsart}
%\usepackage[top = 2cm, bottom = 1cm, left = 1cm, right = 1cm]{geometry}
\usepackage{exsheets, tasks}
\usepackage{amsmath, amssymb, amsthm} %standard AMS packages
\usepackage{marginnote} %marginnotes
\usepackage{gensymb} %miscellaneous symbols
\usepackage{commath} %differential symbols
\usepackage{xcolor} %colours
\usepackage{cancel} %cancelling terms
\usepackage[free-standing-units, space-before-unit]{siunitx} %formatting units
\usepackage{tikz, pgfplots} %diagrams
\usetikzlibrary{calc, hobby, patterns, intersections, decorations.markings}
\usepackage{graphicx} %inserting graphics
\usepackage{hyperref} %hyperlinks
\usepackage{datetime} %date and time
\usepackage{ulem} %underline for \emph{}
\usepackage{xfrac} %inline fractions
\usepackage{enumerate,enumitem} %numbered lists
\usepackage{float} %inserting floats
\usepackage{circuitikz}[american voltages, american currents] %circuit diagrams

\newcommand\numberthis{\addtocounter{equation}{1}\tag{\theequation}} %adds numbers to specific equations in non-numbered list of equations

\newcommand{\AxisRotator}[1][rotate=0]{
	\tikz [x=0.25cm,y=0.60cm,line width=.2ex,-stealth,#1] \draw (0,0) arc (-150:150:1 and 1);%
} %rotation symbols on axes

\theoremstyle{definition}
\newtheorem{example}{Example}
\newtheorem{definition}{Definition}

\theoremstyle{theorem}
\newtheorem{theorem}{Theorem}

\newcommand{\curl}{\mathrm{curl\,}}

\makeatletter
\@addtoreset{section}{part} %resets section numbers in new part
\makeatother

\renewcommand{\thesubsection}{(\arabic{subsection})}
\renewcommand{\thesection}{(\arabic{section})}

%section headings on left
\makeatletter
\def\specialsection{\@startsection{section}{1}%
	\z@{\linespacing\@plus\linespacing}{.5\linespacing}%
	%  {\normalfont\centering}}% DELETED
	{\normalfont}}% NEW
\def\section{\@startsection{section}{1}%
	\z@{.7\linespacing\@plus\linespacing}{.5\linespacing}%
	%  {\normalfont\scshape\centering}}% DELETED
	{\normalfont\scshape}}% NEW
\makeatother

%forces newline after subsection
\makeatletter
\def\subsection{\@startsection{subsection}{3}%
	\z@{.5\linespacing\@plus.7\linespacing}{.1\linespacing}%
	{\normalfont\itshape}}
\makeatother

\settasks{counter-format = tsk[1].}

\SetupExSheets{solution/print = true}

%opening
\title{Introduction to Linear Systems : Assignment 1}
\author
{
	Aakash Jog\\
	ID : 989323563
}
\date{\formatdate{22}{10}{2015}}

\begin{document}

\tikzset{->-/.style={decoration={
  markings,
  mark=at position #1 with {\arrow{>}}},postaction={decorate}}}

\maketitle
%\setlength{\mathindent}{0pt}

\begin{question}
	Consider the following system:
	\begin{align*}
		y^{(2)}(t) + 3 y^{(1)}(t) + 2 y(t) & = u^{(1)}(t) + 3 u(t) \\
		y(0^-)                             & = 1                   \\
		y^{(1)}(0^-)                       & = 1
	\end{align*}
	\begin{enumerate}
		\item
			Find the homogeneous solution.
		\item
			Find the system's impulse response, $g(t)$.
		\item
			Find the system's total response for the input $f(t) = \sin(2 t) \delta_{-1}(t)$.
			Hint: $\int\limits_{0}^{t} e^{a \tau} \sin(b \tau) \dif \tau = \frac{e^{a t} \left( a \sin(b t) - b \cos(b t) \right)}{a^2 + b^2} + \frac{b}{a^2 + b^2}$.
		\item
			Find the particular solution for the input $f(t) = \left( 4 \sin(2 t) + \frac{1}{2} \cos(2 t) \right) \delta_{-1}(t)$.
	\end{enumerate}
\end{question}

\begin{solution}
	\begin{enumerate}[leftmargin = *]
		\item
			The corresponding homogeneous ODE is
			\begin{align*}
				y''(t) + 3 y'(t) + 2 y(t) & = 0
			\end{align*}
			Therefore, the characteristic equation is
			\begin{align*}
				\lambda^2 + 3 \lambda + 2 & = 0
			\end{align*}
			Therefore,
			\begin{align*}
				\lambda & = \frac{-3 \pm \sqrt{9 - 8}}{2}
			\end{align*}
			Therefore,
			\begin{align*}
				\lambda_1 & = -1 \\
				\lambda_2 & = -2
			\end{align*}
			Therefore,
			\begin{align*}
				y_1 & = e^{-t} \\
				y_2 & = e^{-2 t}
			\end{align*}
			Therefore, the homogeneous solution is
			\begin{align*}
				y_h(t) & = A e^{-t} + B e^{-2 t}
			\end{align*}
			Therefore,
			\begin{align*}
				{y_h}'(t) & = -A e^{-t} - 2 B e^{-2 t}
			\end{align*}
			Substituting the initial conditions,
			\begin{align*}
				1 & = A + B \\
				1 & = -A - 2 B
			\end{align*}
			Therefore, solving,
			\begin{align*}
				A & = 3 \\
				B & = -2
			\end{align*}
			Therefore,
			\begin{align*}
				y_h(t) & = \left( 3 e^{-t} - 2 e^{-2 t} \right) \delta_{-1}(t)
			\end{align*}
		\item
			The corresponding homogeneous ODE is
			\begin{align*}
				y''(t) + 3 y'(t) + 2 y(t) & = 0
			\end{align*}
			Therefore, the characteristic equation is
			\begin{align*}
				\lambda^2 + 3 \lambda + 2 & = 0
			\end{align*}
			Therefore,
			\begin{align*}
				\lambda & = \frac{-3 \pm \sqrt{9 - 8}}{2}
			\end{align*}
			Therefore,
			\begin{align*}
				\lambda_1 & = -1 \\
				\lambda_2 & = -2
			\end{align*}
			Therefore,
			\begin{align*}
				y_1 & = e^{-t} \\
				y_2 & = e^{-2 t}
			\end{align*}
			Therefore, the homogeneous solution is
			\begin{align*}
				y_{\delta} & = A e^{-t} + B e^{-2 t}
			\end{align*}
			Therefore,
			\begin{align*}
				{y_{\delta}}' & = -A e^{-t} - 2 B e^{-2 t}
			\end{align*}
			As the input is a delta function, the initial conditions are
			\begin{align*}
				y_{\delta}(0^+)         & = 0 \\
				{y_{\delta}}^{(1)}(0^+) & = 1
			\end{align*}
			Therefore, substituting,
			\begin{align*}
				0 & = A + B \\
				1 & = -A - 2 B
			\end{align*}
			Therefore, solving,
			\begin{align*}
				A & = 1 \\
				B & = -1
			\end{align*}
			Therefore,
			\begin{align*}
				y_{\delta}(t)               & = e^{-t} - e^{-2 t} \\
				\therefore {y_{\delta}}'(t) & = -e^{-t} + 2 e^{-2 t}
			\end{align*}
			Therefore,
			\begin{align*}
				g(t) & = {y_{\delta}}'(t) + 3 y_{\delta}(t) \\
                                     & = \left( 2 e^{-t} - e^{-2 t} \right) \delta_{-1}(t)
			\end{align*}
		\item
			\begin{align*}
				y_p(t) & = g(t) \ast f(t)                                                                                                                                                 \\
                                       & = \int\limits_{-\infty}^{\infty} g(t - \tau) f(\tau) \dif \tau                                                                                                   \\
                                       & = \int\limits_{0}^{t} \left( 2 e^{-(t - \tau)} - e^{-2(t - \tau)} \right) \sin(2 \tau) \dif \tau                                                                 \\
                                       & = 2 e^{-t} \int\limits_{0}^{t} e^{\tau} \sin(2 \tau) \dif \tau - e^{-2 t} \int\limits_{0}^{t} e^{2 \tau} \sin(2 \tau) \dif \tau                                  \\
                                       & = \quad 2 e^{-t} \left( \frac{e^t \left( \sin(2 t) - 2 \cos(2 t) \right)}{1^2 + 2^2} + \frac{2}{1^2 + 2^2} \right)                                               \\
                                       & \quad - e^{-2 t} \left( \frac{e^{2 t} \left( 2 \sin(2 t) - 2 \cos(2 t) \right)}{2^2 + 2^2} + \frac{2}{2^2 + 2^2} \right)                                         \\
                                       & = \frac{2}{5} \left( \sin(2 t) - 2 \cos(2 t) + \frac{4}{5}e^{-t} \right) - \left( \frac{1}{4}\left( \sin(2 t) - \cos(2 t) \right) + \frac{1}{4} e^{-2 t} \right) \\
                                       & = \left( \frac{3}{20} \sin(2 t) - \frac{11}{20} \cos(2 t) + \frac{4}{5} e^{-t} - \frac{1}{4} e^{-2 t} \right)
			\end{align*}
			Therefore, the particular solution is
			\begin{align*}
				y_p(t) & = \left( \frac{3}{20} \sin(2 t) - \frac{11}{20} \cos(2 t) + \frac{4}{5} e^{-t} - \frac{1}{4} e^{-2 t} \right) \delta_{-1}(t)
			\end{align*}
			Therefore, the total solution is
			\begin{align*}
				y(t) & = y_h(t) + y_p(t)                                                                                                                                                                  \\
                                     & = \left( 3 e^{-t} - 2 e^{-2 t} \right) \delta_{-1}(t) + \left( \frac{3}{20} \sin(2 t) - \frac{11}{50} \cos(2 t) + \frac{4}{5} e^{-t} - \frac{1}{4} e^{-2 t} \right) \delta_{-1}(t) \\
                                     & = \left( \frac{3}{20} \sin(2 t) - \frac{11}{20} \cos(2 t) + \frac{19}{5} e^{-t} - \frac{9}{4} e^{-2 t} \right) \delta_{-1}(t)
			\end{align*}
		\item
			\begin{align*}
				f(t) & = 4 \sin(2 t) + \frac{1}{2} \cos(2 t) \\
                                     & = 4 \sin(2 t) + \frac{1}{4} {\sin(2 t)}'
			\end{align*}
			Let $y_{p_1}$ be the particular solution for the input $\sin(2 t)$.\\
			Therefore,
			\begin{align*}
				y_{p_1}(t)               & = \left( \frac{3}{20} \sin(2 t) - \frac{11}{20} \cos(2 t) + \frac{4}{5} e^{-t} - \frac{1}{4} e^{2 t} \right) \delta_{-1}(t) \\
				\therefore {y_{p_1}}'(t) & = \left( \frac{3}{10} \cos(2 t) + \frac{11}{20} \sin(2 t) - \frac{4}{5} e^{-t} + \frac{1}{2} e^{-2 t} \right)
			\end{align*}
			Let $y_{p_2}$ be the particular solution for the input $f(t) = 4 \sin(2 t) + \frac{1}{2} \cos(2 t)$.\\
			Therefore,
			\begin{align*}
				y_{p_2}(t) & = 4 y_{p_1}(t) + \frac{1}{4} {y_{p_1}}'(t)                                                                                      \\
                                           & = \quad 4 \left( \frac{3}{20} \sin(2 t) - \frac{11}{20} \cos(2 t) + \frac{4}{5} e^{-t} - \frac{1}{4}e^{-2 t} \right)            \\
                                           & \quad + \frac{1}{4} \left( \frac{3}{10} \cos(2 t) + \frac{11}{20} \sin(2 t) - \frac{4}{5} e^{-t} + \frac{1}{2} e^{-2 t} \right) \\
                                           & = \frac{7}{8} \sin(2 t) - \frac{17}{8} \cos(2 t) + 3 e^{-t} - \frac{7}{8} e^{-2 t}
			\end{align*}
			Therefore, the particular solution is
			\begin{align*}
				y_{p_2}(t) & = \frac{7}{8} \sin(2 t) - \frac{17}{8} \cos(2 t) + 3 e^{-t} - \frac{7}{8} e^{-2 t}
			\end{align*}
			Therefore, the total solution is
			\begin{align*}
				y(t) & = y_h + y_{p_2}(t)                                                                                                                                                     \\
                                     & = \left( 3 e^{-t} - 2 e^{-2 t} \right) \delta_{-1}(t) + \left( \frac{7}{8} \sin(2 t) - \frac{17}{8} \cos(2 t) + 3 e^{-t} - \frac{7}{8} e^{-2 t} \right) \delta_{-1}(t) \\
                                     & = \left( \frac{7}{8} \sin(2 t) - \frac{17}{8} \cos(2 t) + 6 e^{-t} - \frac{23}{8} e^{-2 t} \right) \delta_{-1}(t)
			\end{align*}
	\end{enumerate}
\end{solution}

\begin{question}
	A linear system is given by the following ODE
	\begin{align*}
		y^{(2)}(t) + a y^{(1)}(t) + b y(t) & = 3 u^{(1)}(t) + 5 u(t)
	\end{align*}
	It is known that the system's response to the initial conditions' response is
	\begin{align*}
		y_h(t) & = \left( 2 e^{-t} + 4 e^{-5 t} \right) \delta_{-1}(t)
	\end{align*}
	\begin{enumerate}
		\item
			Find the values of $a$ and $b$.
		\item
			What are the system's initial conditions at $t = 0^-$?
		\item
			Irrespective of what you found so far, assume from here and on that $a = 6$, $b = 5$, and that the initial conditions are 
			\begin{align*}
				y(0^-)       & = 0 \\
				y^{(1)}(0^-) & = 1
			\end{align*}
			Find the system's total response to an impulse input.
		\item
			Find the system's total response for a unit step input.
	\end{enumerate}
\end{question}

\begin{solution}
	\begin{enumerate}[leftmargin=*]
		\item
			As the homogeneous solution is
			\begin{align*}
				y_h(t) & = \left( 2 e^{-t} + 4 e^{-5 t} \right) \delta_{-1}(t)
			\end{align*}
			the roots of the characteristic equation are
			\begin{align*}
				\lambda_1 & = -1 \\
				\lambda_2 & = -5
			\end{align*}
			Therefore, the characteristic equation is
			\begin{align*}
				(\lambda - \lambda_1) (\lambda - \lambda_2) & = 0 \\
				\therefore (\lambda + 1) (\lambda + 5)      & = 0 \\
				\therefore \lambda^2 + 6 \lambda + 5        & = 0
			\end{align*}
			Therefore, the homogeneous ODE is
			\begin{align*}
				y^{(2)}(t) + 6 y^{(1)}(t) + 5 y(t) & = 0
			\end{align*}
			Therefore, comparing,
			\begin{align*}
				a & = 6 \\
				b & = 5
			\end{align*}
		\item
			The system's inital conditions' response is
			\begin{align*}
				y_h(t)               & = \left( 2 e^{-t} + 4 e^{-5 t} \right) \delta_{-1}(t) \\
				\therefore {y_h}'(t) & = \left( -2 e^{-t} - 20 e^{-5 t} \right) \delta_{-1}(t)
			\end{align*}
			Therefore, substituting the initial conditions,
			\begin{align*}
				y_h(0^+)         & = y_h(0^-)         \\
                                                 & = 2 + 4            \\
                                                 & = 6                \\
				{y_h}^{(1)}(0^+) & = {y_h}^{(1)}(0^-) \\
                                                 & = -2 - 20          \\
                                                 & = -22
			\end{align*}
		\item
			The homogeneous ODE is
			\begin{align*}
				y^{(2)}(t) + 6 y^{(1)}(t) + 5 y(t) & = 0
			\end{align*}
			Therefore,
			\begin{align*}
				\lambda_1 & = -1 \\
				\lambda_2 & = -5
			\end{align*}
			Therefore,
			\begin{align*}
				y_{\delta}(t)                    & = A e^{-t} + B e^{-5 t} \\
				\therefore {y_{\delta}}^{(1)}(t) & = -A e^{-t} - 5 B e^{-5 t}
			\end{align*}
			As the input is a delta function
			\begin{align*}
				y_{\delta}(0)         & = 0 \\
				{y_{\delta}}^{(1)}(0) & = 1
			\end{align*}
			Therefore, substituting
			\begin{align*}
				0 & = A + B \\
				1 & = -A - 5 B
			\end{align*}
			Therefore, solving
			\begin{align*}
				A & = \frac{1}{4} \\
				B & = -\frac{1}{4}
			\end{align*}
			Therefore,
			\begin{align*}
				y_{\delta}(t)                    & = \frac{1}{4} e^{-t} - \frac{1}{4} e^{-5 t} \\
				\therefore {y_{\delta}}^{(1)}(t) & = -\frac{1}{4} e^{-t} + \frac{5}{4} e^{-5 t}
			\end{align*}
			Therefore,
			\begin{align*}
				g(t) & = 3 {y_{\delta}}^{(1)}(t) + 5 y_{\delta}(t)                                               \\
                                     & = -\frac{3}{4} e^{-t} + \frac{15}{4} e^{-5 t} + \frac{5}{4} e^{-t} - \frac{5}{4} e^{-5 t} \\
                                     & = \frac{1}{2} e^{-t} + \frac{5}{2} e^{-5 t}
			\end{align*}
		\item
			\begin{align*}
				y_h(t)                    & = A e^{-t} + B e^{-5 t} \\
				\therefore {y_h}^{(1)}(t) & = -A e^{-t} - 5 B e^{-5 t}
			\end{align*}
			Therefore, substituting the initial values,
			\begin{align*}
				0 & = A + B \\
				1 & = -A - 5 B
			\end{align*}
			Therefore, solving,
			\begin{align*}
				A & = \frac{1}{4} \\
				B & = -\frac{1}{4}
			\end{align*}
			Therefore, the homogeneous solution is
			\begin{align*}
				y_h(t) & = \left( \frac{1}{4} e^{-t} - \frac{1}{4} e^{-5 t} \right) \delta_{-1}(t)
			\end{align*}
			The system's impulse response is
			\begin{align*}
				g(t) & = \left( \frac{1}{2} e^{-t} + \frac{5}{2} e^{-5 t} \right) \delta_{-1}(t)
			\end{align*}
			Therefore,
			\begin{align*}
				y_p(t) & = g(t) \ast u(t)                                                                               \\
                                       & = \int\limits_{0}^{t} \left( \frac{1}{2} e^{-\tau} + \frac{5}{2} e^{-5 \tau} \right) \dif \tau \\
                                       & = \left. -\frac{1}{2} e^{-t} \right|_{0}^{t} - \left. \frac{1}{2} e^{-5 t} \right|_{0}^{t}     \\
                                       & = -\frac{1}{2} \left( e^{-t} - 1 \right) - \frac{1}{2} \left( e^{-5 t} - 1 \right)             \\
                                       & = -\frac{1}{2} e^{-t} - \frac{1}{2} e^{-5 t} + 1
			\end{align*}
			Therefore, the particular solution is
			\begin{align*}
				y_p(t) & = \left( -\frac{1}{2} e^{-t} - \frac{1}{2} e^{-5 t} + 1 \right) \delta_{-1}(t)
			\end{align*}
			Therefore, the total solution is
			\begin{align*}
				y(t) & = y_h(t) + y_p(t)                                                                                                                                        \\
                                     & = \left( \frac{1}{4} e^{-t} - \frac{1}{4} e^{-5 t} \right) \delta_{-1}(t) + \left( -\frac{1}{2} e^{-t} - \frac{1}{2} e^{-5 t} + 1 \right) \delta_{-1}(t) \\
                                     & = \left( -\frac{1}{4} e^{-t} - \frac{3}{4} e^{-5 t} + 1 \right) \delta_{-1}(t)
			\end{align*}
	\end{enumerate}
\end{solution}

\begin{question}
	Consider the system given by
	\begin{align*}
		y^{(2)}(t) + 6 y^{(1)}(t) + 5 y(t) & = u^{(1)}(t) + 4 u(t) \\
		y(0^-)                             & = 1                   \\
		y^{(1)}(0^-)                       & = -1
	\end{align*}
	\begin{enumerate}
		\item
			Find the homogeneous solution.
		\item
			Find the system's impulse response.
		\item
			Find the system's total response for the input $f(t) = \left( e^{-t} + 5 e^{-2 t} \right) \delta_{-1}(t)$.
		\item
			It is given that the system's total response for the input $f(t) = \left( A e^{-t} + B e^{-4 t} \right) \delta_{-1}(t)$ is $y(t) = e^{-t} \delta_{-1}(t)$.
			Find the coefficients $A$ and $B$.
	\end{enumerate}
\end{question}

\begin{solution}
	\begin{enumerate}[leftmargin=*]
		\item
			\begin{align*}
				y^{(2)}(t) + 6 y^{(1)}(t) + 5 y(t) & = u^{(1)}(t) + 4 u(t)
			\end{align*}
			Therefore, the corresponding homogeneous ODE is
			\begin{align*}
				y^{(2)}(t) + 6 y^{(1)}(t) + 5 y(t) & = 0
			\end{align*}
			Therefore, the characteristic equation is
			\begin{align*}
				\lambda^2 + 6 \lambda + 5 & = 0
			\end{align*}
			Therefore,
			\begin{align*}
				\lambda_1 & = -1 \\
				\lambda_2 & = -5
			\end{align*}
			Therefore,
			\begin{align*}
				y_1 & = e^{-t} \\
				y_2 & = e^{-5 t}
			\end{align*}
			Therefore, the homogeneous solution is
			\begin{align*}
				y_h(t)               & = A e^{-t} + B e^{-5 t} \\
				\therefore {y_h}'(t) & = -A e^{-t} - 5 B e^{-5 t}
			\end{align*}
			Substituting the initial conditions,
			\begin{align*}
				1  & = A + B \\
				-1 & = -A - 5 B
			\end{align*}
			Therefore, solving,
			\begin{align*}
				A & = 1 \\
				B & = 0
			\end{align*}
			Therefore,
			\begin{align*}
				y_h(t) & = e^{-t} \delta_{-1}(t)
			\end{align*}
		\item
			\begin{align*}
				y^{(2)}(t) + 6 y^{(1)}(t) + 5 y(t) & = u^{(1)}(t) + 4 u(t)
			\end{align*}
			Therefore, the corresponding homogeneous ODE is
			\begin{align*}
				y^{(2)}(t) + 6 y^{(1)}(t) + 5 y(t) & = 0
			\end{align*}
			Therefore, the characteristic equation is
			\begin{align*}
				\lambda^2 + 6 \lambda + 5 & = 0
			\end{align*}
			Therefore,
			\begin{align*}
				\lambda_1 & = -1 \\
				\lambda_2 & = -5
			\end{align*}
			Therefore,
			\begin{align*}
				y_1 & = e^{-t} \\
				y_2 & = e^{-5 t}
			\end{align*}
			Therefore, the homogeneous solution is
			\begin{align*}
				y_h(t)               & = A e^{-t} + B e^{-5 t} \\
				\therefore {y_h}'(t) & = -A e^{-t} - 5 B e^{-5 t}
			\end{align*}
			As the input is a delta function,
			\begin{align*}
				y_{\delta}(0^+)    & = 0 \\
				{y_{\delta}}'(0^+) & = 1
			\end{align*}
			Therefore, substituting,
			\begin{align*}
				0 & = A + B \\
				1 & = -A - 5 B
			\end{align*}
			Therefore, solving,
			\begin{align*}
				A & = \frac{1}{4} \\
				B & = -\frac{1}{4}
			\end{align*}
			Therefore,
			\begin{align*}
				y_{\delta}(t)              & = \left( \frac{1}{4} e^{-t} - \frac{1}{4} e^{-5 t} \right) \delta_{-1}(t) \\
				\therefore {y_{\delta}}(t) & = \left( -\frac{1}{4} e^{-t} + \frac{5}{4} e^{-5 t} \right) \delta_{-1}(t)
			\end{align*}
			Therefore,
			\begin{align*}
				g(t) & = {y_{\delta}}'(t) + 4 y_{\delta}(t)                                                                                     \\
                                     & = \left( -\frac{1}{4} e^{-t} + \frac{5}{4} e^{-5 t} \right) + 4 \left( \frac{1}{4} e^{-t} - \frac{1}{4} e^{-5 t} \right) \\
                                     & = \frac{3}{4} e^{-t} + \frac{1}{4} e^{-5 t}
			\end{align*}
			Therefore, the system's total impulse response is
			\begin{align*}
				g(t) & = \left( \frac{3}{4} e^{-t} + \frac{1}{4} e^{-5 t} \right) \delta_{-1}(t)
			\end{align*}
		\item
			\begin{align*}
				y_p(t) & = g(t) \ast u(t)                                                                                                                                                                           \\
                                       & = \int\limits_{0}^{t} \left( \frac{3}{4} e^{-(t - \tau)} + \frac{1}{4} e^{-5(t - \tau)} \right) \left( e^{-\tau} + 5 e^{-2 \tau} \right) \dif \tau                                         \\
                                       & = \frac{3}{4} e^{-t} \int\limits_{0}^{t} \left( 1 + 5 e^{-t} \right) \dif \tau + \frac{1}{4} e^{-5 t} \int\limits_{0}^{t} \left( e^{4 \tau} + 5 e^{3 \tau} \right) \dif \tau               \\
                                       & = \frac{3}{4} e^{-t} \left. \left( \tau - 5 e^{-\tau} \right) \right|_{0}^{t} + \frac{1}{4} e^{-5 t} \left. \left( \frac{1}{4} e^{4 \tau} + \frac{5}{3} e^{3 \tau} \right) \right|_{0}^{t} \\
                                       & = \frac{3}{4} e^{-t} \left( t - 5 e^{-t} + 5 \right) + \frac{1}{4} e^{-5 t} \left( \frac{1}{4} e^{4 t} + \frac{5}{3} e^{3 t} - \left( \frac{1}{4} + \frac{5}{3} \right) \right)            \\
                                       & = \frac{61}{16} e^{-t} + \frac{3}{4} t e^{-t} - \frac{10}{3} e^{-2 t} - \frac{23}{48} e^{-5 t}
			\end{align*}
			Therefore, the particular solution is
			\begin{align*}
				y_p(t) & = \left( \frac{61}{16} e^{-t} + \frac{3}{4} t e^{-t} - \frac{10}{3} e^{-2 t} - \frac{23}{48} e^{-5 t} \right) \delta_{-1}(t)
			\end{align*}
			Therefore, the total solution is
			\begin{align*}
				y(t) & = y_h(t) + y_p(t) \\
                                     & = \left( \frac{77}{16} e^{-t} + \frac{3}{4} t e^{-t} - \frac{10}{3} e^{-2 t} - \frac{23}{48} e^{-5 t} \right) \delta_{-1}(t)
			\end{align*}
		\item
			\begin{align*}
				y(t)                             & = y_h(t) + y_p(t) \\
				\therefore e^{-t} \delta_{-1}(t) & = e^{-t} \delta_{-1}(t) + y_p(t)
			\end{align*}
			Therefore,
			\begin{align*}
				y_p(t)       & = g(t) \ast u(t)                                                                                                                                                                             \\
				\therefore 0 & = \int\limits_{0}^{t} \left( \frac{3}{4} e^{-t} + \frac{1}{4} e^{-5 t} \right) \left( A e^{-(t - \tau)} + B e^{-4 (t - \tau)} \right) \dif \tau                                              \\
                                             & = A e^{-t} \int\limits_{0}^{t} \left( \frac{3}{4} + \frac{1}{4} e^{-4 \tau} \right) + B e^{-4 t} \int\limits_{0}^{t} \left( \frac{3}{4} e^{3 \tau} + \frac{1}{4} e^{-\tau} \right) \dif \tau \\
                                             & = A e^{-t} \left. \left( \frac{3}{4} \tau - \frac{1}{16} e^{-4 t} \right) \right|_{0}^{t} + B e^{-4 t} \left. \left( \frac{1}{4} e^{3 \tau} - \frac{1}{4} e^{- \tau} \right) \right|_{0}^{t} \\
                                             & = A e^{-t} \left( \frac{3}{4} t - \frac{1}{16} e^{-4 t} + \frac{1}{16} \right) + B e^{-4 t} \left( \frac{1}{4} e^{3 t} - \frac{1}{4} e^{-t} \right)                                          \\
                                             & = A \left( \frac{3}{4} t e^{-t} - \frac{1}{16} e^{-5 t} + \frac{1}{16} e^{-t} \right) + B \left( \frac{1}{4} e^{-t} - \frac{1}{4} e^{-5 t} \right)                                           \\
                                             & = e^{-t} \left( \frac{1}{16} A \right) + t e^{-t} \left( \frac{3}{4} A \right) + e^{-5 t} \left( -\frac{1}{16} A - \frac{1}{4} B \right)
			\end{align*}
			Therefore,
			\begin{align*}
				\frac{1}{16} A                  & = 0 \\
				\frac{3}{4} A                   & = 0 \\
				-\frac{1}{16} A - \frac{1}{4} B & = 0
			\end{align*}
			Therefore, solving,
			\begin{align*}
				A & = 0 \\
				B & = 0
			\end{align*}
	\end{enumerate}
\end{solution}

\begin{question}
	It is given that the impulse response of system 1 is
	\begin{align*}
		g_1(t) & = \left( e^{-t} + e^{-2 t} \right) \delta_{-1}(t)
	\end{align*}
	It is given that system 1 is of minimal order, i.e. a minimal order system that satisfies the requirements.
	\begin{enumerate}
		\item
			Find the ODE that represents system 1.
		\item
			It is now given that the initial conditions of system 1 are
			\begin{align*}
				y_1(0^-)         & = 2 \\
				{y_1}^{(1)}(0^-) & = 1
			\end{align*}
			Find the unit step response of system 1.
	\end{enumerate}
	Another system, system 2, is given.
	It is known that the particular solution of system 2 to a unit step input is
	\begin{align*}
		{y_p}_2(t) & = \left( -2 + e^{-t} + e^{-4 t} + t e^{-4 t} \right) \delta_{-1}(t)
	\end{align*}
	It is given that system 2 is of minimal order.
	\begin{enumerate}[resume]
		\item
			Find the impulse response of system 2, $g_2(t)$.
		\item
			The initial conditions of system 2 are given to be
			\begin{align*}
				y_2(0^-)         & = 2 \\
				{y_1}^{(1)}(0^-) & = 3 \\
				{y_1}^{(2)}(0^-) & = 1
			\end{align*}
			Find the homogeneous solution of system 2.
	\end{enumerate}
\end{question}

\begin{solution}
	\begin{enumerate}[leftmargin=*]
		\item
			\begin{align*}
				g_1(t) & = \left( e^{-t} + e^{-2 t} \right) \delta_{-1}(t)
			\end{align*}
			Therefore,
			\begin{align*}
				\lambda_1 & = -1 \\
				\lambda_2 & = -2
			\end{align*}
			Therefore, the characteristic equation is
			\begin{align*}
				(\lambda - \lambda_1) (\lambda - \lambda_2) & = 0 \\
				\therefore \lambda^2 + 3 \lambda + 2        & = 0
			\end{align*}
			Therefore, the corresponding homogeneous ODE is
			\begin{align*}
				{y_1}''(t) + 3 {y_1}'(t) + 2 y_1(t) & = 0
			\end{align*}
			Therefore, the ODE is
			\begin{align*}
				{y_1}''(t) + 3 {y_1}'(t) + 2 y_1(t) & = b_1 {u_1}'(t) + b_0 u(t)
			\end{align*}
			For $y_{\delta}$, the initial conditions are
			\begin{align*}
				y_{\delta}(0^+)    & = 0 \\
				{y_{\delta}}'(0^+) & = 1
			\end{align*}
			Therefore, solving for $y_{\delta}$,
			\begin{align*}
				{y_{\delta}}''(t) + 3 {y_{\delta}}'(t) + 2 y_{\delta}(t) & = 0 \\
			\end{align*}
			Therefore,
			\begin{align*}
				y_{\delta}(t)    & = A e^{-t} + B e^{-2 t} \\
				{y_{\delta}}'(t) & = -A e^{-t} - 2 B e^{-2 t}
			\end{align*}
			Therefore, substituting the initial conditions,
			\begin{align*}
				0 & = A + B \\
				1 & = -A - 2 B
			\end{align*}
			Therefore,
			\begin{align*}
				A & = 1 \\
				B & = -1
			\end{align*}
			Therefore,
			\begin{align*}
				y_{\delta}(t)               & = e^{-t} - e^{-2 t} \\
				\therefore {y_{\delta}}'(t) & = -e^{-t} + 2 e^{-2 t}
			\end{align*}
			As $g_1(t)$ is the impulse response,
			\begin{align*}
				g_1(t) & = b_1 {y_{\delta}}'(t) + b_0 y_{\delta}(t)                                       \\
                                       & = b_1 \left( -e^{-t} + 2 e^{-2 t} \right) + b_0 \left( e^{-t} - e^{-2 t} \right) \\
                                       & = e^{-t} (b_0 - b_1) + e^{-2 t} (2 b_1 - b_0)
			\end{align*}
			Therefore, comparing with the given impulse response,
			\begin{align*}
				b_0 - b_1   & = 1 \\
				2 b_1 - b_0 & = 1
			\end{align*}
			Therefore,
			\begin{align*}
				b_0 & = 3 \\
				b_1 & = 2
			\end{align*}
			Therefore, the ODE is
			\begin{align*}
				{y_1}''(t) + 3 {y_1}'(t) + 2 y_1(t) & = 2 {u_1}'(t) + 3 u(t)
			\end{align*}
		\item
			\begin{align*}
				{y_1}''(t) + 3 {y_1}'(t) + 2 y_1(t) & = 2 {u_1}'(t) + 3 u(t) \\
				y_1(0^-)                            & = 2                    \\
				{y_1}'(0^-)                         & = 1
			\end{align*}
			Therefore, the homogeneous ODE is
			\begin{align*}
				{y_1}''(t) + 3 {y_1}'(t) + 2 y_1(t) & = 0 \\
			\end{align*}
			Therefore, the characteristic equation is
			\begin{align*}
				\lambda^2 + 3 \lambda + 2 & = 0
			\end{align*}
			Therefore,
			\begin{align*}
				\lambda_1 & = -1 \\
				\lambda_2 & = -2
			\end{align*}
			Therefore,
			\begin{align*}
				{y_h}_1(t)               & = A e^{-t} + B e^{-2 t} \\
				\therefore {{y_h}_1}'(t) & = -A e^{-t} - 2 B e^{-2 t}
			\end{align*}
			Substituting the initial conditions,
			\begin{align*}
				2 & = A + B \\
				1 & = -A - 2 B
			\end{align*}
			Therefore, solving,
			\begin{align*}
				A & = 5 \\
				B & = -3
			\end{align*}
			Therefore,
			\begin{align*}
				{y_h}_1(t) & = 5 e^{-t} - 3 e^{-2 t}
			\end{align*}
			As $g_1(t)$ is the impulse response,
			\begin{align*}
				{y_p}_1(t) & = g_1(t) \ast u(t)                                                     \\
                                           & = \int\limits_{0}^{t} g_1(t) u(t - \tau)                               \\
                                           & = \int\limits_{0}^{t} \left( e^{-\tau} + e^{-2 \tau} \right) \dif \tau \\
                                           & = -e^{-t} - \frac{1}{2} e^{-2 t} + \frac{3}{2}
			\end{align*}
			Therefore,
			\begin{align*}
				y_1(t) & = {y_h}_1(t) + {y_p}_1(t) \\
                                       & = \left( e 3^{-t} - \frac{7}{2} e^{-2 t} + \frac{3}{2} \right) \delta_{-1}(t)
			\end{align*}
		\item
			\begin{align*}
				{y_p}_2(t) & = \left( -2 + e^{-t} + e^{-4 t} + t e^{-4 t} \right) \delta_{-1}(t)
			\end{align*}
			Therefore, as the impulse response is the derivative of the particular solution part of the step response,
			\begin{align*}
				g_2(t) & = \dod{}{t}\left( -2 + e^{-t} + e^{-4 t} + t e^{-4 t} \right) \delta_{-1}(t) \\
                                       & = \left( -e^{-t} - 3 e^{-4 t} - 4 t e^{-4 t} \right) \delta_{-1}(t)
			\end{align*}
		\item
			\begin{align*}
				g_2(t) & = \left( -e^{-t} - 3 e^{-4 t} - 4 t e^{-4 t} \right) \delta_{-1}(t)
			\end{align*}
			Therefore,
			\begin{align*}
				\lambda_1 & = -1 \\
				\lambda_2 & = -4 \\
				\lambda_3 & = -4
			\end{align*}
			Therefore, the characteristic equation is
			\begin{align*}
				(\lambda + 1) (\lambda + 4)^2                        & = 0 \\
				\therefore \lambda^3 + 9 \lambda^2 + 24 \lambda + 16 & = 0
			\end{align*}
			Therefore, the homogeneous ODE is
			\begin{align*}
				y'''(t) + 9 y''(t) + 24 y'(t) + 16 y(t) & = 0
			\end{align*}
			Therefore, the homogeneous solution is
			\begin{align*}
				y_h(t)                & = A e^{-t} + B e^{-4 t} + C t e^{-4 t}            \\
				\therefore {y_h}'(t)  & = -A e^{-t} + (C - 4 B) e^{-4 t} - 4 C t e^{-4 t} \\
				\therefore {y_h}''(t) & = A e^{-t} + (16 B - 8 C) e^{-4 t} + 16 t e^{-4 t}
			\end{align*}
			Substituting the initial conditions,
			\begin{align*}
				2 & = A + B        \\
				3 & = -A + C - 4 B \\
				1 & = A + 16 B - 8 C
			\end{align*}
			Therefore, solving,
			\begin{align*}
				A & = \frac{19}{3}  \\
				B & = -\frac{13}{3} \\
				C & = -8
			\end{align*}
			Therefore, the homogeneous solution is
			\begin{align*}
				{y_h}_2(t) & = \left( \frac{19}{3} e^{-t} - \frac{13}{3} e^{-4 t} - 8 t e^{-4 t} \right) \delta_{-1}(t)
			\end{align*}
	\end{enumerate}
\end{solution}

\end{document}
